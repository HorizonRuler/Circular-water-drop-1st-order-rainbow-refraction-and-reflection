\documentclass[12pt]{scrartcl}
\setlength{\marginparwidth}{2cm}
\usepackage{tikz}
\usepackage{pgfmath}
\usetikzlibrary{intersections, scopes, angles, quotes}
\newcommand{\pgfextractangle}[3]{%
    \pgfmathanglebetweenpoints{\pgfpointanchor{#2}{center}}
                              {\pgfpointanchor{#3}{center}}
    \global\let#1\pgfmathresult  
}
\begin{document}

% Tunable parameters!
\def \indexofrefraction {1.33}
\def \heightabovehorizontal {6.08cm}
\def \radiusofcircle {7.05cm}

\title{1st Order Rainbow}
\author{Mingyuan Wang}
\subtitle{Exit angle as a function of entering height, index of refraction, and drop diameter for a circular water drop}
\maketitle
\begin{figure}[h]
    \centering
    \begin{tikzpicture}
        \fill (0, 0) circle (2pt) node (O) {} node [anchor = south] {O};
        \draw [name path = circle, blue] (O) circle (\radiusofcircle);
        \path [name path = line1] (-\radiusofcircle, \heightabovehorizontal) coordinate(I) -- (0, \heightabovehorizontal);
        \draw [red, name intersections = {of = circle and line1}] (I) -- (intersection-1);
        \filldraw (intersection-1) circle (2pt) node (A) {} node [anchor = north] {A};
        \pgfextractangle{\anglei}{A}{O}
        \def \angler {asin(sin(360 - \anglei) / \indexofrefraction))};
        \path [name path = line2] (A) -- ++({-360 + \anglei + \angler} : 2 * \radiusofcircle);
        \draw [red, name intersections = {of = circle and line2}] (A) -- (intersection-1);
        \filldraw (intersection-1) circle (2pt) node (B) {} node[anchor = west] {B};
        \path [name path = line3] (B) -- ++({180 + (-360 + \anglei + \angler) + 2 * \angler} : 2 * \radiusofcircle);
        \draw [red, name intersections = {of = circle and line3}] (B) -- (intersection-1);
        \filldraw (intersection-1) circle (2pt) node (C) {} node [anchor = west] {C};
        \pgfextractangle{\angleo}{C}{O}
        \draw [red] (C) -- ++({180 + \angleo - (360 - \anglei)} : 1.5cm) coordinate(C'');
        {[dashed]
            \draw (O) -- node [above right] {r} (A) -- ([turn]0:1.5cm) coordinate(A');
            \draw (O) -- node [above] {r} (B) -- ([turn]0:1.5cm) coordinate(B');
            \draw (O) -- node [right] {r} (C) -- ([turn]0:1.5cm) coordinate(C');
            \draw (C) -- ++(1.5cm, 0);
            \draw (C) -- ++(-1.5cm, 0) coordinate(C''');
            \draw (-\radiusofcircle, 0) coordinate(O') -- (\radiusofcircle, 0);
            \draw[<->] (O') -- node [above right] {y} (I);
        }
        \filldraw (O') circle (2pt) node (O') {} node [above right] {O'};
        \path pic["$\alpha$",angle radius = 1cm,angle eccentricity=1.5, draw] {angle=O--A--B};
        \path pic["$\alpha$",angle radius = 1cm,angle eccentricity=1.5, draw] {angle=A--B--O};
        \path pic["$\alpha$",angle radius = 1cm,angle eccentricity=1.5, draw] {angle=O--B--C};
        \path pic["$\alpha$",angle radius = 1cm,angle eccentricity=1.5, draw] {angle=B--C--O};
        \path pic["$\beta$",angle eccentricity=1.5, draw] {angle=A'--A--I};
        \path pic["$\beta$",angle eccentricity=1.5, draw] {angle=C''--C--C'};
        \path pic["$\beta$",angle eccentricity=1.5, draw] {angle=A--O--O'};
        \path pic["$4\alpha$",angle radius = 1cm,angle eccentricity=1.5, draw] {angle=A--O--C};
        \path pic["$\theta$",angle radius = 1cm,angle eccentricity=1.5, draw] {angle=C'''--C--C''};
    \end{tikzpicture}
\end{figure}
\end{document}